% Created 2020-04-12 dom 20:16
% Intended LaTeX compiler: pdflatex
\documentclass[a4paper]{article}

\usepackage{booktabs}
\usepackage[margin=2cm]{geometry}
\usepackage{amsmath,amsfonts,amssymb,amsthm}
\usepackage{sourcecodepro}
\usepackage[utf8]{inputenc}
\usepackage{booktabs}
\usepackage{array}
\usepackage{colortbl}
\usepackage{listings}
\usepackage{algpseudocode}
\usepackage{algorithm}
\usepackage{graphicx}
\usepackage[english, ]{babel}
\usepackage[scale=2]{ccicons}
\usepackage{hyperref}
\usepackage{relsize}
\usepackage{amsmath}
\usepackage{bm}
\usepackage{amsfonts}
\usepackage{wasysym}
\usepackage{float}
\usepackage{ragged2e}
\usepackage{textcomp}
\usepackage{pgfplots}
\usepackage{todonotes}
\usepgfplotslibrary{dateplot}
\lstdefinelanguage{ein-julia}%
{morekeywords={abstract,struct,break,case,catch,const,continue,do,else,elseif,%
end,export,false,for,function,immutable,mutable,using,import,importall,if,in,%
macro,module,quote,return,switch,true,try,catch,type,typealias,%
while,<:,+,-,::,/},%
sensitive=true,%
alsoother={$},%
morecomment=[l]\#,%
morecomment=[n]{\#=}{=\#},%
morestring=[s]{"}{"},%
morestring=[m]{'}{'},%
}[keywords,comments,strings]%
\lstset{ %
backgroundcolor={},
basicstyle=\ttfamily\scriptsize,
breakatwhitespace=true,
breaklines=true,
captionpos=n,
extendedchars=true,
frame=n,
language=R,
rulecolor=\color{black},
showspaces=false,
showstringspaces=false,
showtabs=false,
stepnumber=2,
stringstyle=\color{gray},
tabsize=2,
}
\renewcommand*{\UrlFont}{\ttfamily\smaller\relax}
\author{Alfredo Goldman}
\date{\today}
\title{Alfredo's MAC0110 Journal}
\hypersetup{
 pdfauthor={Alfredo Goldman},
 pdftitle={Alfredo's MAC0110 Journal},
 pdfkeywords={},
 pdfsubject={},
 pdfcreator={Emacs 26.3 (Org mode 9.3.6)},
 pdflang={Bt-Br}}
\begin{document}

\maketitle

\subsubsection{Agora na linha de comando}
\label{sec:org0684843}
  Hoje, vamos continuar fazendo exercícios com Julia, a diferença é que agora vamos usar
a linha de comando, isso é, vamos criar arquivos fora do ambiente "seguro" do Jupyter.
Para isso, vamos criar arquivos com a extensão .jl com código. Vamos começar criando um
arquivo imprime.jl com apenas um print

\lstset{language=ein-julia,label= ,caption= ,captionpos=b,numbers=none}
\begin{lstlisting}
println("Agora do arquivo")
\end{lstlisting}

Ele pode ser compilado/executado chamando se julia imprime.jl

Da mesma forma podemos usar arquivos para guardar funções como a
de cálculo de potências inteiras, que recebe um valor x, e devolve \(x^ n\).

\lstset{language=ein-julia,label= ,caption= ,captionpos=b,numbers=none}
\begin{lstlisting}
function pot(x, n)
  res = 1
  while n > 0
    res = res * x
    n = n - 1
  end
  return res
end
\end{lstlisting}

 Mas, como podemos usar essa função, agora que ela está pronta? Usando o comando
include, há formas mais sofisticadas de usar "pacotes", mas por enquanto esse comando
será suficiente.

\lstset{language=ein-julia,label= ,caption= ,captionpos=b,numbers=none}
\begin{lstlisting}
include("funct.jl")
println("2 elevado a 4 eh ", pot(2, 4)
\end{lstlisting}

Isso inclusive nos ajuda no que se refere aos testes automatizados, pois o
arquivo com os testes automatizados pode ser executado de forma independente.

Sim, podemos incluir novas funções no arquivo funct.jl, como o cálculo de
fatorial.

Os testes de potência a fatorial podem ser independentes!

Agora que temos as funções de cálculo de potência e de fatorial, podemos usá-las
para cálculos mais sofisticados como o de cosseno, usando séries de Taylor:
\[ \mbox{cos}(x) = \sum_{n = 0}^{\infty} \frac{(-1)^n x^{2n}}{(2n)!} \], sendo que o valor de \(x\) é
dado em radianos.

\lstset{language=ein-julia,label= ,caption= ,captionpos=b,numbers=none}
\begin{lstlisting}
include("funct.jl")
function cosseno(x)
  Erro = 1f-7
  serie = 0
  termo = 1
  i = 0
  while abs(termo > Erro)
     serie = serie + termo
     i = i + 1
     termo = potencia(-1, i) * potencia(x, 2 * i) / fat(2 * i)
  end
  serie = serie + termo
  return serie
end
\end{lstlisting}

Mas, podemos melhorar a nossa função de cálculo de cosseno observando que
dá para a partir do termo anterior, chegar ao próximo termo.


\lstset{language=ein-julia,label= ,caption= ,captionpos=b,numbers=none}
\begin{lstlisting}
# aqui vai o codigo
\end{lstlisting}

Agora uma discussão sobre uma dúvida que apareceu para o monitor, a identação. Isso
é o recuo que fazemos para delimitar blocos de código em Julia. Vamos ver um exemplo:

\lstset{language=ein-julia,label= ,caption= ,captionpos=b,numbers=none}
\begin{lstlisting}
function matematica()
   i = 1
   while i < 100
      if i % 2 == 0
         println(i, " eh par ")
      else
         if i % 3 == 0
            print(i, " eh divisivel por 3 e ")
            soma = 0
            aux = i
            while aux > 0
               soma = soma + aux % 10
               aux = div(aux, 10)
            end
            if soma % 3 == 0
               println(" e a soma dos seus digitos tambem")
            else
               println("Deu ruim, resultado nao esperado para: ", i)
               break # sim esse break e justificado :)
            end
         end
      end
      i = i + 1
   end
end
\end{lstlisting}

Além dos blocos, também é interessante entender o conceito de escopo, vamos a um exemplo:

\lstset{language=ein-julia,label= ,caption= ,captionpos=b,numbers=none}
\begin{lstlisting}
function valeum(a)
   println("a valia: ", a)
   a = 1
   println("agora a vale ", a)
end
function avaleum()
   # nao imprime a aqui, pois daria erro
   a = 1
   println("a vale: ", a)
end

function vamosver()
   a = 3
   println("a vale: ", a)
   begin
     println("Modifiquei a em um bloco")
     a = 2
     println("a vale: ", a)
   end
   println("O a vale, fora do bloco ", a)
   valeum(a)
   println("a vale: ", a)
   avaleum()
   println("a vale: ", a)
end
\end{lstlisting}

 No código acima, podemos ver duas coisas importantes, o escopo (valor de uma variável
vai além dos blocos, pois ao modificar dentro de um bloco, modificamos a variável original.
 Por outro lado, o escopo é independente conforme a função.

 Ao chamar uma função com uma variável é passada uma cópia da variável, que é no início da
função igual ao valor original.

 Um exercício para terminar. Dado um número \(n\) sabe-se que \(n^3\) pode ser representado pela
soma de \(n\) números ímpares consecutivos, encontre esses valores para \(n\) de 1 a 10.
\end{document}
