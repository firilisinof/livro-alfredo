% Created 2020-03-09 seg 20:18
% Intended LaTeX compiler: pdflatex
\documentclass[a4paper]{article}

\usepackage{booktabs}
\usepackage[margin=2cm]{geometry}
\usepackage{amsmath,amsfonts,amssymb,amsthm}
\usepackage{sourcecodepro}
\usepackage[utf8]{inputenc}
\usepackage{booktabs}
\usepackage{array}
\usepackage{colortbl}
\usepackage{listings}
\usepackage{algpseudocode}
\usepackage{algorithm}
\usepackage{graphicx}
\usepackage[english, ]{babel}
\usepackage[scale=2]{ccicons}
\usepackage{hyperref}
\usepackage{relsize}
\usepackage{amsmath}
\usepackage{bm}
\usepackage{amsfonts}
\usepackage{wasysym}
\usepackage{float}
\usepackage{ragged2e}
\usepackage{textcomp}
\usepackage{pgfplots}
\usepackage{todonotes}
\usepgfplotslibrary{dateplot}
\lstdefinelanguage{ein-julia}%
{morekeywords={abstract,struct,break,case,catch,const,continue,do,else,elseif,%
end,export,false,for,function,immutable,mutable,using,import,importall,if,in,%
macro,module,quote,return,switch,true,try,catch,type,typealias,%
while,<:,+,-,::,/},%
sensitive=true,%
alsoother={$},%
morecomment=[l]\#,%
morecomment=[n]{\#=}{=\#},%
morestring=[s]{"}{"},%
morestring=[m]{'}{'},%
}[keywords,comments,strings]%
\lstset{ %
backgroundcolor={},
basicstyle=\ttfamily\scriptsize,
breakatwhitespace=true,
breaklines=true,
captionpos=n,
extendedchars=true,
frame=n,
language=R,
rulecolor=\color{black},
showspaces=false,
showstringspaces=false,
showtabs=false,
stepnumber=2,
stringstyle=\color{gray},
tabsize=2,
}
\renewcommand*{\UrlFont}{\ttfamily\smaller\relax}
\author{Alfredo Goldman}
\date{\today}
\title{Alfredo's MAC0110 Journal}
\hypersetup{
 pdfauthor={Alfredo Goldman},
 pdftitle={Alfredo's MAC0110 Journal},
 pdfkeywords={},
 pdfsubject={},
 pdfcreator={Emacs 26.3 (Org mode 9.3.6)},
 pdflang={Bt-Br}}
\begin{document}

\maketitle


\label{sec:orgab4ed6f}
\subsection{Aula 03 -\textit{[2020-03-09 seg]}}
\label{sec:orgeb07d66}
Objetivo: Ver o interpretador de Julia como uma calculadora poderosa, introduzir a noção de variáveis
\subsubsection{Começando com o modo interativo do Julia.}
\label{sec:org0be9800}
Quem quiser já pode instalar o ambiente de programação, usem esse \href{https://julialang.org/}{link}

Dentro do Julia (após chamar julia na linha de comando), vamos começar com números inteiros:
\lstset{language=ein-julia,label=orgce508e3,caption= ,captionpos=b,numbers=none}
\begin{lstlisting}
1 + 2
\end{lstlisting}

\begin{verbatim}
3
\end{verbatim}


\lstset{language=ein-julia,label= ,caption= ,captionpos=b,numbers=none}
\begin{lstlisting}
40 * 3
\end{lstlisting}

\begin{verbatim}
120
\end{verbatim}


\lstset{language=ein-julia,label= ,caption= ,captionpos=b,numbers=none}
\begin{lstlisting}
84 / 2
\end{lstlisting}
Notem que nesse caso, houve uma mudança de tipos, pois 84 e 2 são inteiros e o resultado
é um número em ponto flutuante (float)
\begin{verbatim}
42.0
\end{verbatim}


Também é possível pedir o resultado inteiro usando o operador div:
\lstset{language=ein-julia,label= ,caption= ,captionpos=b,numbers=none}
\begin{lstlisting}
div(84,2)
\end{lstlisting}

\begin{verbatim}
42
\end{verbatim}


Também dá para fazer a exponenciação:

\lstset{language=ein-julia,label= ,caption= ,captionpos=b,numbers=none}
\begin{lstlisting}
2^31
\end{lstlisting}

\begin{verbatim}
2147483648
\end{verbatim}

Expressões mais complexas também podem ser calculadas:

\lstset{language=ein-julia,label= ,caption= ,captionpos=b,numbers=none}
\begin{lstlisting}
23 + 2 * 2 + 3 * 4
\end{lstlisting}

\begin{verbatim}
39
\end{verbatim}

 Sim, a precedência de operadores usual também é válida em Julia. Mas, ai
vem a primeira lição de programação: * Escreva para humanos, não para máquinas *

\lstset{language=ein-julia,label= ,caption= ,captionpos=b,numbers=none}
\begin{lstlisting}
23 + (2 * 2) + (3 * 4)
\end{lstlisting}

Em julia também podemos fazer operações com números em ponto flutuante:

\lstset{language=ein-julia,label= ,caption= ,captionpos=b,numbers=none}
\begin{lstlisting}
23.5 * 3.14
\end{lstlisting}

\begin{verbatim}
73.79
\end{verbatim}


ou
\lstset{language=ein-julia,label= ,caption= ,captionpos=b,numbers=none}
\begin{lstlisting}
12.5 / 2.0
\end{lstlisting}

\begin{verbatim}
6.25
\end{verbatim}


Acima temos mais um exemplo de código escrito para pessoas, ao se escrever
2.0 estamos deixando claro que o segundo parâmetro é um número float.

É importante saber que números em ponto flutuante tem precisão limitada, logo não se espante com resultados inesperados como abaixo:

\lstset{language=ein-julia,label=org6110caf,caption= ,captionpos=b,numbers=none}
\begin{lstlisting}
1.2 - 1.0
\end{lstlisting}

ou
\lstset{language=ein-julia,label=org676c1ac,caption= ,captionpos=b,numbers=none}
\begin{lstlisting}
0.1 + 0.2
\end{lstlisting}

ou ainda

\lstset{language=ein-julia,label=org39c0b73,caption= ,captionpos=b,numbers=none}
\begin{lstlisting}
10e15 + 1 - 10e15
\end{lstlisting}

Um outro operador interessante é o \% que faz o resto da divisão

\lstset{language=ein-julia,label=orgdf6ac29,caption= ,captionpos=b,numbers=none}
\begin{lstlisting}
4 % 3
\end{lstlisting}

\subsubsection{Variáveis e seus tipos}
\label{sec:org7236f5d}
Em Julia também temos o conceito de variáveis, que servem para armazenar os
diferentes conteúdos de dados possíveis.

\lstset{language=ein-julia,label=org27b9928,caption= ,captionpos=b,numbers=none}
\begin{lstlisting}
a = 7
2 + a
\end{lstlisting}

É importante notar que as variáveis em Julia podem receber novos valores e o tipo
da variável depende do que foi atrubuído inicialmente.

\lstset{language=ein-julia,label=orgf4ba7b6,caption= ,captionpos=b,numbers=none}
\begin{lstlisting}
a = 3
a = a + 1
typeof(a)
\end{lstlisting}

Aproveitando o momento, podemos ver que há vários tipos primitivos em Julia, sendo os
principais:

\lstset{language=ein-julia,label=org0d4629a,caption= ,captionpos=b,numbers=none}
\begin{lstlisting}
typeof(1)
typeof(1.1)
typeof("Bom dia")
\end{lstlisting}

\begin{verbatim}
[....]
\end{verbatim}


Falando em strings, elas são definidas por conjuntos de caracteres entre aspas como:
\lstset{language=ein-julia,label=orgdefba4d,caption= ,captionpos=b,numbers=none}
\begin{lstlisting}
s1 = "Olha que legal"
s2 = "Outra String"
\end{lstlisting}

Dá também para fazer operações como strings como concatenação:

\lstset{language=ein-julia,label=org5871f5c,caption= ,captionpos=b,numbers=none}
\begin{lstlisting}
s1 = "Tenha um"
s2 = " Bom dia"
s3 = s1 * s2
\end{lstlisting}

Ou potência:

\lstset{language=ein-julia,label=orga488a79,caption= ,captionpos=b,numbers=none}
\begin{lstlisting}
s = "Nao vou mais fazer coisas que possam desagradar os meus colegas"
s ^ 10
\end{lstlisting}

Ainda sobre variáveis, há umas regras com relação aos seus nomes, tem que
começar com uma letra, pode ter dígitos e não pode ser uma palavra reservada.  É
bom notar que Julia por ser uma linguagem moderna, aceita nomes de caracteres em
unicode, pode exemplo

\lstset{language=ein-julia,label= ,caption= ,captionpos=b,numbers=none}
\begin{lstlisting}
\delta = 2  # Para se fazer o delta, deve se digitar \ seguido de delta, seguido de <tab>
\end{lstlisting}
\subsubsection{Saída de dados}
\label{sec:org3b9f9f1}
Para fazer saídas usam-se dois comandos, print() e o println(), sendo que o primeiro não pula linha e o segundo pula.
\lstset{language=ein-julia,label=orgbe66d56,caption= ,captionpos=b,numbers=none}
\begin{lstlisting}
print("Hello ")
println("World!")
println("Ola, mundo!")
\end{lstlisting}

Para evitar que se digitem muitos caracteres, por vezes podemos usar "açucares sintáticos".

\lstset{language=ein-julia,label=org91426f9,caption= ,captionpos=b,numbers=none}
\begin{lstlisting}
x = 1
x = x + 1
x += 1  # forma equivalente a acima
\end{lstlisting}


\subsection{Aula 04 -\textit{[2020-03-11 qua]}}
\label{sec:org9fa0164}
Objetivo: Começar a entender como funcionam as funções
\subsubsection{O uso de funções é uma abstração natural}
\label{sec:org9236734}
Na aula passada já vimos umas funções e isso foi bem natural, foram elas:
\begin{itemize}
\item typeof() - Que dado um parâmetro devolve o seu tipo
\item div() - Que dados dois parâmetros devolve a divisão inteira do primeiro pelo segundo
\item print() e println() - Que dados diversos parâmetros os imprime, sem devolver nada
\end{itemize}
Inclusive, aqui vale a pena ver que podemos pedir ajuda ao Julia para saber o que fazem as
funções. Para isso, se usa o ? antes da função:
\lstset{language=ein-julia,label=org2a0d03c,caption= ,captionpos=b,numbers=none}
\begin{lstlisting}
?typeof()
?div()
?print()
\end{lstlisting}

Ao fazer isso, inclusive descobrimos que o div() pode ser usado também como \textdiv{}.

Uma outra função bem útil é a que permite transformar um tipo de valor em outro.

\lstset{language=ein-julia,label=org83ae32e,caption= ,captionpos=b,numbers=none}
\begin{lstlisting}
parse(Float64, "32")
\end{lstlisting}

Para conversão de valores em ponto flutuante para inteiros, temos a função trunc.

\lstset{language=ein-julia,label=org7e0382a,caption= ,captionpos=b,numbers=none}
\begin{lstlisting}
trunc(Int64, 2.25)
\end{lstlisting}

De forma inversa temos o float.

\lstset{language=ein-julia,label=orgf8984f0,caption= ,captionpos=b,numbers=none}
\begin{lstlisting}
float(2)
\end{lstlisting}

Finalmente, podemos transformar um valor em uma string, como em:

\lstset{language=ein-julia,label=org8e112ed,caption= ,captionpos=b,numbers=none}
\begin{lstlisting}
string(3)
\end{lstlisting}
ou
\lstset{language=ein-julia,label=org5401c13,caption= ,captionpos=b,numbers=none}
\begin{lstlisting}
string(3.57)
\end{lstlisting}

Também tem muitas funções matemáticas prontas como
\begin{itemize}
\item sin(x) - calcula  seno de x em radianos
\item cos(x)
\item tan(x)
\item deg2rad(x) - converte x de graus em radianos
\item rad2deg(x)
\item log(x) - calcula o logarítmo natural de x
\item log(x, b) - calcula o logarítmo de x na base b
\item log2(x) - calcula o logarítmo de x na base 2
\item log10(x)
\item exp(x) - calcula o expoente da base natural de x
\item abs(x) - calcula o módulo de x
\item sqrt(x) - calcula a raiz quadrada
\item isqrt(x) - calcula a raiz quadrada inteira de x
\item cbrt(x) - raiz cúbica de x
\item factorial(x) - calcula o fatorial de x
\end{itemize}


\end{document}
