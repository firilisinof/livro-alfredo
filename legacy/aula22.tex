% Created 2020-06-14 dom 22:30
% Intended LaTeX compiler: pdflatex
\documentclass[a4paper]{article}

\usepackage{booktabs}
\usepackage[margin=2cm]{geometry}
\usepackage{amsmath,amsfonts,amssymb,amsthm}
\usepackage{sourcecodepro}
\usepackage[utf8]{inputenc}
\usepackage{booktabs}
\usepackage{array}
\usepackage{colortbl}
\usepackage{listings}
\usepackage{algpseudocode}
\usepackage{algorithm}
\usepackage{graphicx}
\usepackage[english, ]{babel}
\usepackage[scale=2]{ccicons}
\usepackage{hyperref}
\usepackage{relsize}
\usepackage{amsmath}
\usepackage{bm}
\usepackage{amsfonts}
\usepackage{wasysym}
\usepackage{float}
\usepackage{ragged2e}
\usepackage{textcomp}
\usepackage{pgfplots}
\usepackage{todonotes}
\usepgfplotslibrary{dateplot}
\lstdefinelanguage{ein-julia}%
{morekeywords={abstract,struct,break,case,catch,const,continue,do,else,elseif,%
end,export,false,for,function,immutable,mutable,using,import,importall,if,in,%
macro,module,quote,return,switch,true,try,catch,type,typealias,%
while,<:,+,-,::,/},%
sensitive=true,%
alsoother={$},%
morecomment=[l]\#,%
morecomment=[n]{\#=}{=\#},%
morestring=[s]{"}{"},%
morestring=[m]{'}{'},%
}[keywords,comments,strings]%
\lstset{ %
backgroundcolor={},
basicstyle=\ttfamily\scriptsize,
breakatwhitespace=true,
breaklines=true,
captionpos=n,
extendedchars=true,
frame=n,
language=R,
rulecolor=\color{black},
showspaces=false,
showstringspaces=false,
showtabs=false,
stepnumber=2,
stringstyle=\color{gray},
tabsize=2,
}
\renewcommand*{\UrlFont}{\ttfamily\smaller\relax}
\author{Alfredo Goldman}
\date{\today}
\title{Alfredo's MAC0110 Journal}
\hypersetup{
 pdfauthor={Alfredo Goldman},
 pdftitle={Alfredo's MAC0110 Journal},
 pdfkeywords={},
 pdfsubject={},
 pdfcreator={Emacs 26.3 (Org mode 9.3.6)},
 pdflang={Bt-Br}}
\begin{document}

\maketitle

\section{Programa do curso}
\label{sec:org2c5f0a1}
\subsection{Aula 22 - Ainda matrizes}
\label{sec:org9c0ac79}

 Dadas duas matrizes m e n, faça uma função que devolva o
produto delas (sem usar o * para matrizes).

\lstset{language=ein-julia,label= ,caption= ,captionpos=b,numbers=none}
\begin{lstlisting}
function multiplica(a, b)
   dima = size(a)
   dimb = size(b)
   if dima[2] != dimb[1]
     return -1
   end
   c = zeros(dima[1], dimb[2])
   for i in 1:dima[1]
      for j in 1:dimb[2]
         for k in 1:dima[2]
            c[i, j] = c[i, j] + a[i, k] * b[k, j]
         end
       end
   end
   return c
 end

\end{lstlisting}

Uma matriz quadrada de tamanho n é um quadrado latino se em cada linha e coluna
aparecem todos os valores de 1 a n. Faça uma função que dada uma matriz
quadrada verifica se ela é um quadrado latino.

\begin{lstlisting}
#
\end{lstlisting}

Dizemos que uma matriz inteira A nxn  é uma matriz de permutação se em cada linha e em cada
coluna houver n-1 elementos nulos e um único elemento igual a 1. Faça uma função que recebe
uma matriz quadrada e que verifica se ela é uma matriz de permutação.



\begin{lstlisting}
#
\end{lstlisting}

Mas, conforme a abstração que fazemos as matrizes podem representar coisas diferentes,
por exemplo dada uma matriz quadrada n x n, a posição a[i][j], pode indicar se há um caminho
entre a cidade i e a cidade j.

Da mesma forma, a[i][j] pode representar a distância entre a cidade i e a cidade j. Com isso, podemos
ter problemas mais sofisticados como saber se dá para chegar da cidade i na cidade j, ou o custo
do menor caminho.

Nessa aula, vamos ver, um exemplo mais elaborado do uso de matriz, através de uma teoria conhecida
como percolation. Vamos primeiro entender o que seria isso usando o livro do Sedgewick e do Wayne
\url{https://introcs.cs.princeton.edu/java/24percolation/}

A pergunta é saber qual é a probabilidade, a partir da qual a percolation ocorre com grande
chance (digamos mais de 80\%, ou seja em 100 tentativas, em 80 ocorre a percolation). Vamos para simplificar
o problema pensar em matrizes 20 por 20.

Como resolver esse problema?
\end{document}
